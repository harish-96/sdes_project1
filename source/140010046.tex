\documentclass{report}
\usepackage[utf8]{inputenc}
\usepackage{amsmath}
\usepackage{graphicx}
\usepackage{graphics}
\usepackage{parskip}


\title{SDES}
\author{Harish Murali}
\date{October 2016}

\setlength{\parskip}{10pt}
\begin{document}

\maketitle
%\vspace*{-2cm}
\section{Introduction}
This report is a part of the submission for a course project on Van der Pol oscillator in the course on Software Development Tools for Engineers and Scientists. The code to solve the equations is written in Python programming language.
The Van der Pol oscillator is a non-conservative oscillator with non-linear damping. It evolves according to the second order linear differential equation
\begin{equation}
\frac{d^2x}{dt^2} - \mu(1 - x^2)\frac{dx}{dt} + x = 0
\end{equation}
where x is the position coordinate -- which is a function of time t, and $\mu$ is a scalar parameter indicating non linearity in the strength of the damping.

In a Van der Pol oscillator, energy is dissipated at low amplitudes and generated at high amplitudes. Thus, there exists a stable state around which the dissipation an generation of energies balance. It can be proved that for a given $\mu$, every trajectory eventually converges to this stable cycle. 

The Van der Pol equation finds its applications in several fields of physics and biology. It is also used to model geological faults in seismology. It is generally

\section{Results}

The following plots have been obtained solving the differential equation with the initial conditions of %$\mu = 0.5$, $x = 0$ and $\dot{x} = 1$

\begin{align*}
    \mu &= 0.5 \\
    x &= 0 \\
    \dot{x} &= 1
\end{align*}

\begin{figure}[ht]
\includegraphics[width=\textwidth]{"../output/trajectory"}
\caption{} % add caption
\label{trajectory}
\end{figure}
\newpage

The phase space diagram for the Van der Pol oscillator for the same initial conditions is
\begin{figure}[ht]
\includegraphics[width=\textwidth]{"../output/phase_portrait"}
\caption{} % add caption
\label{phaseportrait}
\end{figure}


\section{Installation Instructions}

The following are the dependencies, please refer to the well documented installation instructions for these online:
\begin{itemize}
	\item Python 3 or above
	\item Numpy
	\item Scipy
	\item Matplotlib
\end{itemize}

With these installed, go to the download directory and in the bash terminal, type make.
To edit the parameters of the Van der Pol oscillator, make changes to the input.csv file in the required columns.

\end{document}
